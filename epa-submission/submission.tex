\documentclass{article}
\usepackage[utf8]{inputenc}
\usepackage[margin=1in]{geometry}

\usepackage{graphicx}
\graphicspath{{images}}

\usepackage{titlesec}
\titleformat*{\section}{\large\bfseries}
\titleformat*{\subsection}{\normalfont\bfseries}

\usepackage{caption}
\captionsetup[figure]{font={small, bf}}

\usepackage[style=numeric,backend=biber,sorting=none]{biblatex}
\addbibresource{ag_bib.bib}

\usepackage{hyperref}

\title{EPA Response}
\author{\large Amanjit Gill}
\date{\small \today}

\begin{document}

\maketitle

\section{Omission of Pertinent Information}

So the 24-hour max is for the tipping hall, not for the sealed containers waiting outside; how long do the outside containers stay? if a while, does odour and fire risk management take account of extended stays?

what are abnormal operating conditions? is it in emergency, or is it something like an unusually busy day?

how will you know how much of the waste in the mixed industrial truck is putrescible?

\section{Consequential Errors}

mixed up eastern and western

\section{Unsound Case for Site Selection}

\subsection{Linkage to State Government Documents}

Sections 8.1.1 to 8.1.4 reference several state government documents to build an argument for the selection of the Hallam Rd site for the WTS. While these documents make a case for the need for a WTS, they fail to make any case for a WTS located specifically at the Hallam Rd landfill site:

\begin{itemize}
  \item Plan Melbourne 2017-2050 \cite{planmelb}: No mention of Hallam Road landfill.
  \item Recycling Victoria - A New Economy \cite{recyclingvic}: No mention of Hallam Road landfill.
  \item Statewide Waste and Resource Recovery Infrastructure Plan \cite{swrrip}: Section 3.3.2, table 3.2, lists Hallam Road Landfill among the "hubs of state importance". It provides a brief description of current operations and future challenges, suggesting resource recovery as a way to conserve airspace. Activities such as that proposed in the application are not mentioned in this table, or anywhere else, in the context of the Hallam Road site.
  \item Metropolitan Waste and Resource Recovery Implementation Plan \cite{mwrrip}: The introduction to section 6 lists Hallam Road among those sites with capacity challenges. It proposes the diversion of organic waste away from landfill. Section 6.2 suggests resource recovery as a potential future use of closed landfills, without mentioning any specific site. Section 8.1.2, table 15, contains similar text to that quoted from \cite{swrrip}.
\end{itemize}

\subsection{Commercial Argument}

Section 8.1.5 states “it is necessary that the upgraded Transfer Station, which is the subject of this Development Licence Application, be operable by the time construction of the EfW plant is completed in approximately 2026”.

This is an irrelevant argument, because there is no environmental imperative for ensuring the proposed WTS is built to Veolia's preferred timetable. It is not inherently unsafe or risky for the Maryvale EfW (energy from waste) plant to idly wait for waste input. The risk is entirely commercial. 

Veolia is part of the consortium that will deliver the Maryvale plant \cite{maryvaleefw}. This means it can exert control over both the Maryvale and Hampton Park developments. Therefore, if the commercial risk was so serious, it could have located an alternative site for the WTS and started the development earlier instead of waiting for the landfill to close.


\subsection{Linkage to Casey Council Strategic Planning}

In section 8.2.1, the applicant references the council’s Hampton Park Hill Development Plan \cite{hphplan}, stating “the subject site of the proposed Hampton Park WTS is indicated as land for ‘Waste and Resource Recovery’ and as such the use and development of the land for a Waste Transfer Station is consistent with the adopted development plan”.

However, the cited Development Plan does not clarify that a future facility on the site may process putrescible waste; it only discusses putrescible waste in the context of existing landfill operations. 

Despite this, This is notable because it was exhibited for community consultation in 2022 \cite{hphplan}


It repeatedly cites the state government's Hallam Road Waste and Resource Recovery Hub Plan \cite{hubplan}, which briefly mentions the potential to "aggregate and consolidate putrescible kerbside material" at the site. However, this Hub Plan was never exhibited for community consultation, and \cite{caseyhph} never alludes to its contemplations regarding putrescible waste.  \item 

However, the cited Development Plan uses the phrase “waste transfer” only once, with no elaboration other than a requirement in section 4.2 that any new or redeveloped waste transfer facility include an attractive interface to neighbouring land uses, and acoustic shielding. Given the phrase "waste transfer" is almost exclusively 

Therefore, if Veolia intends to suggest it has social licence to develop the proposed WTS, it cannot do so on the basis of the Development Plan.
 
\section{Assertions with no Evidence}

\section{Reliance on Incomplete Consultation}

\section{Fit and Proper}

\section{Misapplication of Planning Rules (if relevant)}



\printbibliography

\end{document}

