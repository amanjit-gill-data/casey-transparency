\documentclass{article}
\usepackage[utf8]{inputenc}
\usepackage[margin=1in]{geometry}

\usepackage{graphicx}
\graphicspath{{images}}

\usepackage{titlesec}
\titleformat*{\section}{\large\bfseries}

\usepackage{caption}
\captionsetup[figure]{font={small, bf}}

\usepackage[style=numeric,backend=biber,sorting=none]{biblatex}
\addbibresource{ag_bib.bib}

\usepackage{hyperref}

\title{EPA Response}
\author{\large Amanjit Gill}
\date{\small \today}

\begin{document}

\maketitle

\section{Omission of Pertinent Information}

So the 24-hour max is for the tipping hall, not for the sealed containers waiting outside; how long do the outside containers stay? if a while, does odour and fire risk management take account of extended stays?

what are abnormal operating conditions? is it in emergency, or is it something like an unusually busy day?

how will you know how much of the waste in the mixed industrial truck is putrescible?

\section{Consequential Errors}

mixed up eastern and western

\section{Fallacious Case for Site Selection}

Sections 8.1.1 to 8.1.4 reference several state government documents to build an argument for the selection of the Hallam Rd site for the WTS.

While these documents make a case for the need for a WTS, they fail to make any case for a WTS \textit{located at the Hallam Rd landfill site}:

\begin{itemize}
  \item Plan Melbourne 2017-2050 \cite{planmelb}: No mention of Hallam Road Landfill.
  \item Recycling Victoria - A New Economy \cite{recyclingvic}: No mention of Hallam Road Landfill.
  \item Statewide Waste and Resource Recovery Infrastructure Plan \cite{swrrip}: Section 3.3.2, Table 3.2, lists Hallam Road Landfill among the "hubs of state importance". It provides a brief description of current operations and future challenges, suggesting resource recovery as a way to conserve airspace. Activities such as that proposed in the application are not mentioned in this table, or anywhere else, in the context of the Hallam Road site.
  \item Metropolitan Waste and Resource Recovery Implementation Plan \cite{mwrrip}: The introduction to Section 6 lists Hallam Road among those sites with capacity challenges. It proposes the diversion of organic waste away from landfill. Section 6.2 suggests resource recovery as a potential future use of closed landfills, without mentioning any specific site. Section 8.1.2, Table 15, contains similar text on the Hallam Road landfill to that found in \cite{swrrip}.
\end{itemize}

in 8.1.5, application states, “it is necessary that the upgraded Transfer Station, which is the subject of this Development Licence Application, be operable by the time construction of the EfW plant is completed in approximately 2026”.

However, it's not inherently unsafe or risky for Maryvale to sit idle waiting for waste input. The only risk is a commercial one, for Veolia, because its EfW will not be generating income. This is not our problem.

Veolia part-owns Maryvale and fully owns HP landfill. it's in control of both projects. so if the commercial risk was so serious, it could have located a site across the 9 council areas and started the development earlier instead of waiting for the landfill to close.


in 8.2.1, application attempts to support the case for the WTS by referencing the council’s Hampton Park Hill Development Plan, stating that “the subject site of the proposed Hampton Park WTS is indicated as land for ‘Waste and Resource Recovery.’ and as such the use and development of the land for a Waste Transfer Station is consistent with the adopted development plan”.

however, the Development Plan uses the phrase “waste transfer station” only once,

\section{Assertions with no Evidence}

\section{Reliance on Incomplete Consultation}

\section{Fit and Proper}

\section{Misapplication of Planning Rules (if relevant)}



\printbibliography

\end{document}

