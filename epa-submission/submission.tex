\documentclass[12pt]{article}
\usepackage[utf8]{inputenc}
\usepackage[margin=1in]{geometry}

\usepackage{titlesec}
\titleformat*{\section}{\large\bfseries}
\titleformat*{\subsection}{\normalfont\bfseries}

\usepackage[style=numeric,backend=biber,sorting=none]{biblatex}
\addbibresource{ag_bib.bib}

\usepackage{hyperref}

\title{\textbf{An Appraisal of EPA Development Licence Application APP032219}}
\author{Amanjit Gill}
\date{\today}

\begin{document}

\maketitle

\section*{Introduction}

Veolia Australia and New Zealand recently applied to the Environment Protection Authority (EPA) for a Development Licence that would allow it to build and operate a waste transfer station (WTS) on the same site as an existing landfill. The application appears to contain several errors, oversights and omissions. 

The purpose of this analysis is to identify and substantiate these deficiencies, to support an argument that Veolia should withdraw, amend and re-submit its application after selecting a more suitable site for development.

Due to constraints on resources and time, this analysis is not exhaustive; it covers a selection of significant deficiencies. 

\section{Unsound Case for Site Selection}

\subsection{Linkage to State Government Documents}

Sections 8.1.1 to 8.1.4 reference several state government documents to build an argument for the selection of the Hallam Rd site for the WTS. While these documents make a case for the need for a WTS, they fail to make any case for a WTS located specifically at the Hallam Rd landfill site:

\begin{itemize}
  \item Plan Melbourne 2017-2050 \cite{planmelb}: No mention of Hallam Road landfill.
  \item Recycling Victoria - A New Economy \cite{recyclingvic}: No mention of Hallam Road landfill.
  \item Statewide Waste and Resource Recovery Infrastructure Plan \cite{swrrip}: Section 3.3.2, table 3.2, lists Hallam Road Landfill among the "hubs of state importance". It provides a brief description of current operations and future challenges, suggesting resource recovery as a way to conserve airspace. Activities such as that proposed in the application are not mentioned in this table, or anywhere else, in the context of the Hallam Road site.
  \item Metropolitan Waste and Resource Recovery Implementation Plan \cite{mwrrip}: The introduction to section 6 lists Hallam Road among those sites with capacity challenges. It proposes the diversion of organic waste away from landfill. Section 6.2 suggests resource recovery as a potential future use of closed landfills, without mentioning any specific site. Section 8.1.2, table 15, contains similar text to that quoted from \cite{swrrip}.
\end{itemize}

\subsection{Commercial Argument}

Section 8.1.5 states “it is necessary that the upgraded Transfer Station, which is the subject of this Development Licence Application, be operable by the time construction of the EfW plant is completed in approximately 2026”.

This is an irrelevant argument, because there is no environmental imperative for ensuring the proposed WTS is built to Veolia's preferred timetable. It is not inherently unsafe or risky for the Maryvale EfW (energy from waste) plant to idly wait for waste input. The risk is entirely commercial. 

Veolia is part of the consortium that will deliver the Maryvale plant \cite{maryvaleefw}. This means it can exert control over both the Maryvale and Hampton Park developments. Therefore, if the commercial risk was so serious, it could have located an alternative site for the WTS and started the development earlier instead of waiting for the landfill to close.

\subsection{Linkage to Casey Council Strategic Planning}

In section 8.2.1, the applicant references the council’s Hampton Park Hill Development Plan \cite{caseyhph}, stating “the subject site of the proposed Hampton Park WTS is indicated as land for ‘Waste and Resource Recovery’ and as such the use and development of the land for a Waste Transfer Station is consistent with the adopted development plan”.

However, the cited Development Plan does not clarify that a future facility on the site may process putrescible waste; it only discusses putrescible waste in the context of existing landfill operations. 

The Development Plan cites the state government's Hallam Road Waste and Resource Recovery Hub Plan \cite{hubplan}, which briefly mentions the potential to "aggregate and consolidate putrescible kerbside material" at the site. Unlike \cite{caseyhph}, this Hub Plan was never exhibited for community consultation, and the Development Plan never alludes to its contemplations regarding putrescible waste. 

In addition, the Development Plan uses the phrase “waste transfer” only once, with no elaboration other than the amenity-related requirements listed in section 4.2. Given there is already a waste transfer station on the site \cite{outlook}, and all 265 transfer stations in Victoria are "resource recovery centres" with various purposes \cite{vicwastemap}, one cannot infer from the Development Plan alone that the Hallam Road site may be earmarked for large-scale compaction of putrescible waste.  

Therefore, if Veolia intends to suggest prior community awareness and acceptance of a WTS at the Hallam Road landfill site, it cannot do so on the basis of the Development Plan.
 
\subsection{Lack of Options Analysis}

Not only has the applicant failed to mount a substantive case for locating the proposed WTS at the Hallam Road landfill site, it has also failed to give due consideration to alternative sites. This is notable because the proposed facility will service nine municipalities \cite{semawp}; it is unlikely that other suitable sites do not exist.

In order for a major project decision to inspire confidence, it must be tethered to a defensible decision-making process. To this end, Infrastructure Australia provides guidance on Multi-Criteria Decision Analysis \cite{infraus}, the Victorian government offers advice on its IMS (investment management standard) \cite{ims}, and several examples of options assessment are available, such as that for offshore wind transmission \cite{vicgrid}.

In the absence of evidence that some form of methodical decision-making process has occurred, Veolia's argument on site selection lacks credibility.

\section{Incorrect Assumptions}

\subsection{Cumulative Odour Impacts}

Section 9.3 of the application states "the Hallam Road landfill is nearing
capacity and is planned to close, presenting a low risk of cumulative odour impacts". This is the basis upon which Veolia has chosen not to consider odour from the landfill in its argument in favour of a variation to the recommended separation distance. 

This reasoning is fallacious for two reasons:

\begin{itemize}
  \item Landfills can produce odourous gases for many years after they close; the typical aftercare period is 30 years, but this may be longer at sites like Hampton Park where there are filled cells that were built to previous, inferior, standards \cite{hubplan}. Landfill gas is mostly composed of methane and carbon dioxide, but it also contains components that cause the well-known "rotten egg" smell \cite{lfginfo}. 
  \item Section 11.1.4 of the application states that Veolia intends to transfer leachate from the WTS into an existing leachate pond after the landfill closes. Leachate is a potential source of odour \cite{epalandfill}.
\end{itemize}

Any analysis of cumulative effects from the existing landfill would be incomplete without the inclusion of these sources of odour. 

\subsection{Spurious Comparison with Banksmeadow}

The applicant has used field odour surveillance results from Banksmeadow, New South Wales, on the basis that this facility is operationally similar to the proposed WTS. However, the two facilities cannot validly be compared for this purpose. Banksmeadow TT (Transfer Terminal) transfers its leachate to tanks for eventual offsite disposal \cite{banksmeadow}, whereas the proposed Hampton Park WTS will rely on an existing open-air pond. 

The use of tanks at Banksmeadow may be a significant mitigation of its odour risk, and may cause the odour risk assessment for the proposed WTS on Hallam Rd to be overly optimistic. 

\section{Omission of Pertinent Details}

The body of the application document leaves several questions unanswered. It is difficult to make a determination on the merit of the application without this information.

\begin{enumerate}
  \item The 24-hour maximum holding time applies to the tipping hall; it is not clear that it also applies to the sealed containers waiting outside to be transported to Maryvale. How long might these external containers stay? 
  \item What constitutes abnormal operating conditions? Would it be an emergency, or a more mundane event like an unusually busy day?
  \item If there is to be little or no sorting of waste, does this mean each incoming truck will exclusively contain waste from a single stream?
  \item Without sorting, how will Veolia determine how much of the waste in the mixed industrial truck is putrescible?
\end{enumerate}

\section*{Conclusion}

This analysis shows that Veolia, in its application to the EPA for a licence to develop a WTS at the Hallam Road landfill site:

\begin{itemize}
  \item Mounted a site selection argument that is unsound.
  \item Based its odour risk analysis on incorrect assumptions.
  \item Omitted pertinent operational details.
\end{itemize}

It is difficult to see how the EPA can issue a development licence on the basis of an accumulation of errors, oversights and omissions. A realistic appraisal of the potential impacts on the local community and the environment is not possible unless Veolia withdraws, amends and re-submits its application. 

Given the comprehensive nature of the application's deficiencies, it will likely be difficult for Veolia to make a defensible case for situating the development in Hampton Park; as such, it would be better to pursue alternative sites across the proposed nine-council service area.

\printbibliography

\end{document}

